
\documentclass[t]{beamer}
\usepackage[utf8]{inputenc}
\setbeamercovered{transparent}

% ---- font settings -------------------------------
\usepackage[T1]{fontenc}
\usepackage{tgpagella}
\usefonttheme{serif}
% --------------------------------------------------

% ---- tikz settings -------------------------------
\usepackage{tikz}
\usetikzlibrary{external}
\tikzexternalize[prefix=build/]
\usepackage{tikzscale}
\usepackage[most]{tcolorbox}
\tcbuselibrary{listings}
\tcbset{
  shield externalize,
  dot/.style={
    sharp corners,
    frame hidden,
    top=4pt,left=4pt,right=4pt,bottom=4pt,
    enhanced,
    colback=white,
    coltitle=black,
    fonttitle=\scriptsize,
    attach boxed title to bottom right={
      xshift=-2mm,yshift=2.75mm},
    boxed title style={colback=white,frame hidden, 
      top=0pt,left=0pt,right=0pt,bottom=0pt},
    borderline={.4pt}{0pt}{black,densely dotted},
    segmentation style={line width=.4pt,black,densely dotted},
    }
}
\newcommand{\inptikz}[1]{%
  \tikzsetnextfilename{#1}%
  \input{tikz/#1.tikz}%
}
% --------------------------------------------------

\usepackage{graphicx}
\usepackage[frak=euler]{mathalpha}
\usepackage{mathtools}
\usepackage{hyperref}
\usepackage{booktabs}
\usepackage{biblatex}
\usepackage{chemfig}
\addbibresource{references.bib}
\usepackage{pgfplots}
\pgfplotsset{compat=1.18}
\usepackage{theme/beamerthemeflux}
\usepackage{theme/palettepaultol}
\author{Any Author}
\date{Jan. 1st, 2023}
\title{Beamer Flux Theme}
\institute{Any University}

\begin{document}

\newtcolorbox{dotbox}[1][]{dot,#1}
\newtcblisting{dotlst}[1][]{
  dot,listing only,before=\begin{center},after=\end{center},
  listing options={basicstyle=\ttfamily\scriptsize,commentstyle=\color{gray}\ttfamily,
  language=TeX},
  #1
}

\begin{frame}[plain]
    \maketitle
\end{frame}

\section{Design}

\begin{frame}[fragile]{Basic Design}
\begin{itemize}
  \item Pinciples: \begin{itemize}[label={\tiny\faCircleO}]
    \item Avoid too many colors;
    \item Dotted lines for separation;
  \end{itemize}
  \item Minimalistic outline on top w. page numbers;
  \item <2-> Expandable boxes for explanation; \only<2>{
    \begin{dotbox}[title=Some elaboration,hbox]
      This box only appears on this slide.
    \end{dotbox}
  }
  \item <3-> References in footnotes.\footcite{1932_OnsagerFuoss}
\end{itemize}
\end{frame}

\begin{frame}{Layout - {\ttfamily\textbackslash{}vdivide} and {\ttfamily\textbackslash{}hdivide}}
\end{frame}

\begin{frame}[fragile]{Default Packages}
Some pacakges are loaded by default, and their styles customized:
\begin{itemize}
    \item enumitem
    \item mathtools
    \item tikz/pgf
    \item chemfig \begin{dotlst}[%
      listing side text,righthand ratio=.48,
      title={Chemfig Example: $\mathrm{[BMIM][HOAc]}$}
    ]
\chemfig{%
  H_3C-[:-36]N**5(---N(-H-%
  [:0,1.5,,,,densely dotted]%
  O-[1]C(=[3]O)-[0]CH_3)--)}
    \end{dotlst}
\end{itemize}
\end{frame}

\section{Options}
\begin{frame}{Theme Options}
  % WIP
\end{frame}

\begin{frame}[fragile]{Plots}
  \begin{columns}[T]
    \column{.48\textwidth}
    \begin{itemize}
      \item Palette taken from \href{https://personal.sron.nl/~pault/#sec:qualitative}{Paul Tol};
      \item Insert plots with \href{https://ctan.org/pkg/tikzscale?lang=en}{tikzscale};
        \begin{dotlst}[title={Example plot}]
\includegraphics[%
  width=\textwidth,%
  height=.75\textwidth]%
  {tikz/randplot.tikz}
        \end{dotlst}
    \end{itemize}
    \column{.48\textwidth}
    \includegraphics[width=\textwidth,height=.75\textwidth]{tikz/randplot.tikz}
  \end{columns}
\end{frame}

\begin{frame}{Length Definitions}
  % WIP
\end{frame}

\appendix
\section*{}

\begin{frame}[fragile]{Random Tips}
\begin{itemize}
  \item Appendices are backup slides (separate page no.);
  \item Externalizing TikZ pictures;%
  \begin{dotlst}[%
    listing side comment,righthand ratio=0.38,comment={\inptikz{onsager}}
    ]
\usepackage{tikz}
\usepackage[most]{tcolorbox}
\usetikzlibrary{external}
\tikzexternalize[prefix=build/]
\tcbset{shield externalize}
\newcommand{\inptikz}[1]{%
  \tikzsetnextfilename{#1}%
  \input{tikz/#1.tikz}%
}
  \end{dotlst}
\end{itemize}
\end{frame}

\end{document}
